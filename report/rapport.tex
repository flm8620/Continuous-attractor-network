\documentclass{article}
\usepackage[utf8]{inputenc}
\usepackage{graphicx}
\usepackage{amsmath}
\usepackage{amsfonts}
\usepackage{amsbsy}
\usepackage{mathrsfs}
\usepackage{appendix}
\usepackage{amsthm}
\usepackage{bbold}
\usepackage{epstopdf}
\usepackage{stmaryrd}
\usepackage[]{algorithm2e}
\usepackage{hyperref}

\title{Reading report for continuous attractor network}
\author{Leman FENG\\ Email: flm8620@gmail.com\\Website: lemanfeng.com}

\begin{document}
	\maketitle
	\section{Background}
	This paper \cite{romani2010continuous} goes directly to the study of a specific case of continuous attractor network without much introduction to basic concepts. So I read other materials about continuous attractor networks before going through the paper.
	
	\subsection{Hopfield network}
	We have seen Hopfield network in the course. In one Hopfield network, many neural units $\{x_i\}_i$ exists. Each unit can only take binary values. The connection strength between each pair of units $w_{ij}$ is symmetrical.
	
	The update rule is simply:
	\begin{equation}
	x_i \leftarrow
	\begin{cases}
	+1 & \text{if} \sum_{j}w_{ij}x_j \geq \theta_i\\
	-1 & \text{otherwise}
	\end{cases}
	\end{equation}
	
	\subsection{Attractor network}
	Attractor network is a general concept. We will often see definition like: An attractor network consists of $N$ neural units connected such that the global dynamics becomes stable in a $D$ dimensional space and $N>>D$.
	
	This implies that attractor network deal with continuous values and continuous update rules. Or more direct, real-valued nodes and differential equations for update rules. So all possible states of network is $\mathbb{R}^N$. And all stable states form a lower dimensional manifold in $\mathbb{R}^N$. Stable states are called attractors. According to the shape of the manifold, there are points, line, ring, plane attractors and so on.
	
	So from my point of view, the attractor network is just the dynamics of a set of real-valued variables ruled by a set of differential equations. It's more like a pure mathematical objects which can be applied in many domains, such as physics, the study of pendulum's dynamics. And biological interpretation is just another possibility.
	
	For biological interpretation, one natural way to construct an attractor network is to extend the Hopfield network in a continuous version. So for each node the update rule is
	\begin{equation}
	\frac{\mathrm{d}x_i}{\mathrm{d}t} = -\lambda x_i + f(\sum_j w_{ij} x_j)
	\end{equation}
	where $f$ is the transfer function.
	
	From this update rule, we can see that without connection from other units, the activity of one neural unit will decay exponentially to zero.
	
	\subsection{Continuous attractor network}
	Then we finally reach the continuous attractor network. The first we need thing to understand is what does it mean by "continuous". Because attractor network deals already with continuous values and differential equations.
	
	From the name, it seems that it talks about a continuous set of attractor. But we already have the line, ring and plane attractors in a normal attractor network. So this must not be the meaning of "continuous". The formal definition of a continuous attractor is very abstract, including minimal closed set of attractors which satisfy certain properties, according to \cite{CANN}. While I didn't find any formal definition for Continuous attractor network, it seems that the only way to have a continuous attractor is to have infinite neural units in network \cite{trappenberg2003continuous}. And the infinite neural units form a continuum of neurons. This continuum of neurons doesn't come from the definition, but from the result of the definition of continuous attractor network.
	
	Since we talks about infinite neural units, it makes no sense to talk about one neural unit. We can only define neural units as a density function $\nu(x)$ over a manifold $\mathbf{X}$. It's just like the difference between discrete and continuous probability distribution. And the weights between neurons is no longer a matrix, but a function defined on the Cartesian product of $\mathbf{X}$ : $W:\mathbf{X}\times \mathbf{X} \rightarrow \mathbb{R} $. 
	
	We use $\mathbf{m}:\mathbf{X}\rightarrow\mathbb{R}$ to denote the activity of network. Then the update rule becomes:
	\begin{equation}
	\frac{\mathrm{d}\mathbf{m}(x)}{\mathrm{d}t} = -\lambda \mathbf{m}(x) + f(\int_{\mathbf{X}}\nu(x)W(x,y)m(y)\mathrm{d}y)
	\end{equation}
	
	\section{Network structure}
	After the study of basic concepts, I dig into this paper \cite{romani2010continuous}. The introduction section is very abstract and I can hardly understand author's motivation about the superposition of multiple maps. So I decide to study the first example with two maps.
	
	The author use the word "map" to refer a single manifold. He defines two circular one-dimensional maps $\Theta_A,\Theta_B$ and the coordinates system are $\theta_A\in[0,2\pi),\theta_B\in[0,2\pi)$. And the neurons live on the Cartesian product of these two maps. And neurons are identified by pair of labels $(\theta_A,\theta_B)$, with their density distribution $\nu(\theta_A,\theta_B)$. The neurons actually live on a surface of torus, circular in both directions.
	
	Despite it makes no sense to say one neuron at coordinate $(\theta_A,\theta_B)$, I will still use this way to express one infinitesimal neighborhood of neuron continuum located near $(\theta_A,\theta_B)$, just as the author does.
	
	The weight between two neurons $(\theta_A,\theta_B)$ and $(\theta_A',\theta_B')$ is given by 
	\begin{equation}
	W((\theta_A,\theta_B),(\theta_A',\theta_B')) = \frac{J_1}{2}[\cos(\theta_A-\theta_A')+\cos(\theta_B-\theta_B')+J_0]
	\end{equation}
	although the author defines $W$ using the expression $W(\theta_A-\theta_A',\theta_B-\theta_B')$, which confused me a lot at the beginning, I still want to write the right definition of weight function to clarify that each neuron is identified by two coordinates, not one.
	
	From this definition, we can see that for neurons close to each other, they have positive interaction between them. For neurons far away from each other, they will have less interaction or even negative interaction, or inhibition, to each other.
	
	And the dynamics is given by 
	\begin{equation}
	\begin{split}
	\tau &\dot{\mathbf{m}}(\theta_A,\theta_B) = -\mathbf{m}(\theta_A,\theta_B) +\\ &\max[0,\int_{\Theta_A\times\Theta_B}\nu(\theta_A',\theta_B')W((\theta_A,\theta_B),(\theta_A',\theta_B'))\mathbf{m}(\theta_A',\theta_B')\mathrm{d}\theta_A'\mathrm{d}\theta_B' + I]
	\end{split}
	\end{equation}
	$I$ indicates the uniform external current.
	
	Until now, we didn't define the density of neurons $\nu(\theta_A,\theta_B)$. The author used a simple way to build a correlated/uncorrelated distribution. By correlation, the author means that we can think about $\nu(\theta_A,\theta_B)$ as a probability distribution (up to a constant), then consider the correlation of two random variable $\nu(\theta_A),\nu(\theta_B)$ which follows the marginal distribution of $\nu(\theta_A,\theta_B)$.
	
	The author used a change of variables. Let's redefine $\theta_A=(\theta-\mu r), \theta_B=(\theta+\mu r)$, modulo $2\pi$, for $(\theta,r)$ uniformly distributed on $\Theta\times R = [0,2\pi)\times[-\pi/2,\pi/2]$, and a parameter $\mu\in[0,1]$.
	
	For me, this definition can be summarized as : let's create uniform neuron continuum over $\Theta_A\times\Theta_B=[0,2\pi)\times[0,2\pi)$, then delete those neurons whose two coordinates are too far apart : the distance between $\theta_A$ and $\theta_B$ over a circular ring is larger than $\mu\pi$.
	
	After this deletion, although the two maps are correlated (if $\mu<1$), the marginal distribution is still uniform over respectively $\Theta_A$ and $\Theta_B$, simply because of the periodic maps and the symmetrical deletion.
	
	\section{Dynamics}
	The author studied the type of attractors in this network. It's hard to study directly the dynamics of the whole network. So the author used the first three Fourier components of network activity : $Z_A,Z_B,\eta$, which is a natural way of simplification because both maps are periodic:
	\begin{equation}
	\begin{split}
	Z_A &= \int_{\Theta_A\times\Theta_B} e^{i\theta_A} \mathbf{m}(\theta_A,\theta_B) v(\theta_A,\theta_B)=\rho_A e^{i\Psi_A}\\
	Z_B &= \int_{\Theta_A\times\Theta_B} e^{i\theta_B} \mathbf{m}(\theta_A,\theta_B) v(\theta_A,\theta_B)=\rho_B e^{i\Psi_B}\\
	\eta &= \int_{\Theta_A\times\Theta_B} \mathbf{m}(\theta_A,\theta_B) v(\theta_A,\theta_B)\\
	\end{split}
	\end{equation}
	$Z_A, Z_B$ are respectively the first Fourier components of marginal activity on $\Theta_A$ and $\Theta_B$, and $\eta$ is just the mean value.
	
	By using again the dynamics of network, we can get the dynamics of these three variables. Notice that $Z_A, Z_B$ are complex numbers, whose modulus $\rho_A,\rho_B$ indicate the strength of peaks in marginal activity, and whose angles $\Psi_A,\Psi_B$ indicate the position of peaks in marginal activity. So in total five degrees of freedom. The author again used a change of variables to introduce dimensionless parameters in function of these five degrees of freedom. The final reduced dynamics are represented by five parameters: $(\gamma, \alpha, \sigma, \Psi_+, \Psi_-)$, with their definitions:
	\begin{equation}
	\begin{split}
	\gamma&=\frac{\rho_B-\rho_A}{\rho_B+\rho_A},\quad \alpha=\frac{J_1}{I}\frac{\rho_B+\rho_A}{2}\\
	\sigma&=\frac{I+J_0\eta}{\alpha I},\quad\\
	\Psi_+ &= \frac{\Psi_A+\Psi_B}{2},\quad \Psi_- = \frac{\Psi_A-\Psi_B}{2}
	\end{split}
	\end{equation}
	The meaning of each parameter is:
	\begin{itemize}
		\item $\gamma\in[-1,1]$, difference of bump magnitude in two maps $\Theta_A,\Theta_B$
		\item $\alpha>0$, scaling factor of the whole activity
		\item $\sigma\in[-1,1]$, spatial size of bump, -1 means no activity and 1 means all neurons are equally active. 
		\item $\Psi_+$, the location of bump in $\Theta$, or the averaged location in $\Theta_A$ and $\Theta_B$, which isn't important because network is periodic on $\Theta$. This parameter is ignore during the study of attractor.
		\item $\Psi_-$, the location of bump in $R$, or the difference of location in $\Theta_A$ and $\Theta_B$. This parameter can only be ignore when $\mu=1$, where the network degenerate to a torus.
	\end{itemize}
	
	\section{Attractors}
	The authors shown that, depend on the value of $J_1$ and $\mu$, there are four kinds of attractors: Homogeneous, Single ring, Double ring and Cylinder. The first Homogeneous case is not interesting. The three latter cases comes from the distinction on value pair $(\gamma,\Psi_-)$. The relation is shown in the table:
	\begin{center}
		\begin{tabular}{ | c | c | c | }
			\hline
			               & $\gamma=0$  & $\gamma\neq 0$ \\ \hline
			$\Psi_-=0$     & Single ring & Double ring \\ \hline
			$\Psi_-\neq 0$ & Cylinder    & Impossible  \\
			\hline
		\end{tabular}
	\end{center}
	\subsection{Simulation}
	To reproduce the result of the three types of attractor, I implemented in Matlab a numerical simulation of the network. The code is available online at \href{https://github.com/flm8620/Continuous-attractor-network}{my github}.I take a discretization of $M\times N$ on the space of $\Theta\times R = [0,2\pi)\times[-\pi/2,\pi/2]$.

	\section{Morph sequence}
	The first part of paper discuss only about the case of two maps.
\bibliographystyle{unsrt}
\bibliography{sample}
\end{document}
